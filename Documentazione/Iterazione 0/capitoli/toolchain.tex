\section{Toolchain}
Per la realizzazione della piattaforma verranno utilizzati i seguenti strumenti:
\begin{itemize}
	\item \textbf{Modellazione}:
	\begin{itemize}
		\item Diagramma dei casi d'uso, deployment diagram, component diagram, class
		diagram, diagrammi di flusso, diagramma entità-relazione: Draw.io e StarUML
	\end{itemize}
	\item \textbf{Implementazione Applicazione Client}:
	\begin{itemize}
		\item Linguaggio di programmazione: Dart;
		\item IDE: Visual Studio Code;
		\item Interfaccia grafica: Flutter;
		\item Software per mockup: Figma;
	\end{itemize}
	\item \textbf{Implementazione Web Server}:
	\begin{itemize}
		\item Linguaggio di programmazione: Java;
		\item IDE: Eclipse;
		\item Framework: Spring;
		\item Analisi statica: STAN4J;
		\item Analisi dinamica: JUnit;
		\item Deployment: Digital Ocean.
	\end{itemize}
	\item \textbf{Implementazione Database}:
	\begin{itemize}
		\item Tipologia: Relazionale;
		\item Database: PosgreSQL;
		\item Provider: Digital Ocean.
	\end{itemize}
	\item \textbf{Documentazione, versioning e organizzazione del team}:
	\begin{itemize}
		\item Documentazione: LaTeX;
		\item Versioning: GitHub;
		\item Git client: Github Desktop;
		\item Organizzazione del Team: Microsoft Teams e Trello.
	\end{itemize}
\end{itemize}