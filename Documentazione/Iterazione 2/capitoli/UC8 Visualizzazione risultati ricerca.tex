\subsection{UC-8 Visualizzazione risultati della ricerca}
Quando vengono trovati uno o più eventi tra quelli in programma viene mostrata una lista di tutti loro.
\\
\\
\textit{Breve Descrizione: L'utente effettua la ricerca di uno o più eventi in base al nome e gli viene mostrata una lista con le corrispondenze} 
\\
\\
\textit{Attori Coinvolti: Sistema, utente}
\\
\\
\textit{Precondizione: L'utente è registrato nell'app}
\\
\\
\textit{Postcondizione: L'utente visualizza una lista dei risultati}
\\
\\
\textit{Procedimento:}
\begin{enumerate}
	\item Cercare un evento così come descritto in UC-7;
	\item La lista degli eventi viene mostrata nella parte inferiore dello schermo;
	\item Se la lista contiene molti elementi può essere fatta scorrere in verticale.
\end{enumerate}


\textit{Eccezioni:}
\begin{itemize}
	\item E1: L'evento con il nome dato non esiste
	\begin{enumerate}
		\item Viene mostrato all'utente un messaggio di errore;
		\item Viene data la possibilità di riprovare o tornare alla homepage;
	\end{enumerate}
\end{itemize}