\subsection{UC-1 Login}
Per l'autenticazione abbiamo utilizzato il framework Spring Security con il metodo ``Basic Access Authentication'': si tratta di una tecnica che non necessita dell'utilizzo di cookie o di mantenere una sessione tra client e server, ma utilizza gli header HTTP per fornire le informazioni di accesso. I campi username e password vengono codificati con base64 e sono poi trasmessi nell'header ogni volta che viene chiamata una API. Il sistema, prima di elaborare una richiesta, verifica che lo username e la password trasmessi appartengano effettivamente ad un utente presente nel database. 
\\
\\
\textit{Breve descrizione:} l'utente compila il form per eseguire il login: in caso di credenziali corrette il sistema consente l'accesso ai servizi, altrimenti notifica l'utente della non correttezza delle credenziali.
\\
\\
\textit{Attori coinvolti:} Utente, sistema.
\\
\\
\textit{Precondizione:} l'utente è registrato nel sistema e apre la app.
\\
\\
\textit{Postcondizione:} l'utente accede alla app (in caso le credenziali siano corrette) oppure viene avvertito che le credenziali sono sbagliate.
\\
\\
\textit{Procedimento:}
\begin{enumerate}
	\item il sistema richiede all'utente le informazioni di accesso: username e password;
	\item l'utente inserisce le informazioni di accesso;
	\item il sistema controlla le informazioni fornite;
	\item le informazioni sono corrette. [E1: le informazioni sono sbagliate].
	\item l'utente viene indirizzato alla homepage dell'applicazione.
\end{enumerate}
\newpage
\textit{Eccezioni:}
\begin{itemize}
	\item E1:
	\begin{enumerate}
		\item le informazioni sono sbagliate;
		\item il sistema comunica all'utente che le informazioni inserite non sono corrette;
		\item ritorno al passo 1 di ``Procedimento''.
	\end{enumerate}
\end{itemize}
