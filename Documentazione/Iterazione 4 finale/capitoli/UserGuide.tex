\section*{Manuale utente}

L'utente può interfacciarsi con l'applicazione tramite diverse schermate.ù

Innanzitutto, e' possibile registrarsi creando un nuovo account o eventualmente eseguire direttamente il login. 
Per fare ciò sono state predisposte le due interfacce \ref{fig:login} e \ref*{fig:signup}.

Eseguito l'accesso tramite le proprie credenziali utente, si viene reindirizzati alla schermata principale che mostra gli eventi
disponibili \ref{fig:home}. In alto a sinistra è possibile accedere al proprio profilo \ref*{fig:profile}, mentre tramite l'apposita
icona si apre la sezione cerca \ref{fig:cerca}, da cui si possono cercare più rapidamente specifici eventi.

La creazione di una nuova escursione avviene facendo click sul bottone "nuova escursione", in quanto si verrà rimandati alla schermata 
da cui inserire le informazioni del nuovo evento \ref*{fig:new}
 (qualora l'utente sia un' organizzatore).

Inoltre, tramite la barra di navigazione è possibile accedere alla sezione dedicata ai consigliati \ref*{fig:suggerimenti} ed alle iscrizioni \ref{fig:iscrizioni}, in cui
vengono mostrati gli eventi ai quali si è iscritti. 

Infine, da qualsiasi schermata è possibile accedere al profilo di ogni evento, nel quale vengono mostrate tutte le specifiche inserite
dall'utente organizzatore.


\begin{figure}[h!]
    \begin{center}
    \begin{multicols}{2}
        \includegraphics[width=0.55\linewidth]{Iterazione 4 finale/images/login.png}
        \caption{Login screen}\label{fig:login}
        \includegraphics[width=0.55\linewidth]{Iterazione 4 finale/images/signup.png}
        \caption{Signup screen}\label{fig:signup}
    \end{multicols}
    \begin{multicols}{2}
        \includegraphics[width=0.55\linewidth]{Iterazione 4 finale/images/profile.png}
        \caption{Profilo utente}\label{fig:profile}
        \includegraphics[width=0.55\linewidth]{Iterazione 4 finale/images/iscrizioni.png}
        \caption{Iscrizioni}\label{fig:iscrizioni}
    \end{multicols}
\end{center}
\end{figure} 

\begin{figure}[h!]
    \begin{center}
    \begin{multicols}{2}
        \includegraphics[width=0.55\linewidth]{Iterazione 4 finale/images/home.png}
        \caption{Home screen}\label{fig:home}
        \includegraphics[width=0.55\linewidth]{Iterazione 4 finale/images/new evento.png}
        \caption{New evento}\label{fig:new}
    \end{multicols}
    \begin{multicols}{2}
        \includegraphics[width=0.55\linewidth]{Iterazione 4 finale/images/suggerimenti.png}
        \caption{Suggerimenti}\label{fig:suggerimenti}
        \includegraphics[width=0.55\linewidth]{Iterazione 4 finale/images/cerca.png}
        \caption{Cerca}\label{fig:cerca}
    \end{multicols}
\end{center}
\end{figure} 